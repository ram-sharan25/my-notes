% Created 2024-10-01 Tue 15:38
% Intended LaTeX compiler: pdflatex
\documentclass[11pt]{article}
\usepackage[utf8]{inputenc}
\usepackage[T1]{fontenc}
\usepackage{graphicx}
\usepackage{longtable}
\usepackage{wrapfig}
\usepackage{rotating}
\usepackage[normalem]{ulem}
\usepackage{amsmath}
\usepackage{amssymb}
\usepackage{capt-of}
\usepackage{hyperref}
\author{Ram Sharan Rimal  }
\date{\today}
\title{Intro to Cloud and Open Stack}
\hypersetup{
 pdfauthor={Ram Sharan Rimal  },
 pdftitle={Intro to Cloud and Open Stack},
 pdfkeywords={},
 pdfsubject={},
 pdfcreator={Emacs 29.4 (Org mode 9.6.15)}, 
 pdflang={English}}
\begin{document}

\maketitle
\tableofcontents


\section{Cloud/Cloud Computing :}
\label{sec:org2491399}
\begin{itemize}
\item Delivery of computing resources- platform , application to the datacenters
on demand as pay per basis over the internet.
\item Instead of owning the private servers companies can rent cloud servers to host
their various services.
\item makes business scalalble, effiecient, and reliable.
\end{itemize}

\subsection{Public Cloud :}
\label{sec:org96a0960}
\begin{itemize}
\item Instead of storing the data and processing them in your local desktop machine
you hire a company provide comuting power and   store data on their powerful
servers and access via the web .
\item This scheme of operation is of public cloud.
\end{itemize}


\subsection{Private Cloud:}
\label{sec:org2e2a400}
\begin{itemize}
\item Instead of having the computation and storage on the employee desk companies
buy big servers where all computation and storage is handled .
\end{itemize}

\subsection{Hybrid Cloud:}
\label{sec:org9f0b3c0}
\begin{itemize}
\item partly public partly private
\item 
\end{itemize}

\subsection{Cloud Service Models:}
\label{sec:orgab7c08f}

\subsubsection{Infrastructure as service:}
\label{sec:orgba2c813}
\begin{itemize}
\item Here cloud provider provides servers, storage, network interfaces, firewalls
as service to the user.
\item Servers can be collection of VM or physical computers
\item Databases stored as block or object storage and  can be accessed directly
or via network drive  or as object via web interface.
\item firewalls for protection of infrastructure is provided.
\item User provides OS image and the application software .
\item User is responsible to main the OS and their softwares
\item Example: Amazon Elastic Compute Cloud (EC2)
\end{itemize}

\subsubsection{Platform as service:}
\label{sec:orga78936a}
\begin{itemize}
\item Cloud service provider provides for complete computing platform along with un
derlying hardware, server,Database ( block or object) , firewalls, network interface
and also OS image. The platform is use ready for the User.
\item User are to provide for the Application software and also responsible to
maintain the application software.
\item Provider is reponsible to main the platform that includes maintaince of servers,
database, maintaining firewalls, also updating the OS Images.
\item Examples: Amazon Web Services, Google App Engine , Elastic , Beanstalk
\end{itemize}

\subsubsection{Software as Service:}
\label{sec:org8877629}
\begin{itemize}
\item Cloud Service Provider provides the hardware infrastructure, server, database,
 ,underlying OS along with the Application installedin it . It is ready to use
for the user.
\item The role of the user is to use th application out of the box.
\item Generally software is not bought but rented.
\item Maintainence of platform and software fall under the responsibility of cloud
provider.
\item Example :
\begin{enumerate}
\item Dropbox
\item Microsoft Office 365
\end{enumerate}
\end{itemize}


\subsubsection{Intermediate Cloud Offerring:}
\label{sec:orge2d1216}
\begin{itemize}
\item Provider which provides SaaS  PaaS ans IaaS operation as servicew offering
\item Salesforce provides platform for developers to add on the main applications
and also salesforce as software offerring.
\item Microsoft Azure provides both IAAS ans PAAS as offerring.
\end{itemize}

\subsubsection{Why Cloud ? Why not Cloud ?}
\label{sec:org38cc042}
\begin{enumerate}
\item Is cloud necessary ?
\item Is it better to stay unclouded ?
\item Is it viable to maintain private cloud ?
\item Should tranfer to public cloud ?
\end{enumerate}

\subsubsection{Scenarios :}
\label{sec:org6e948d0}
\begin{itemize}
\item Public Cloud Advantages :
\begin{itemize}
\item at starting of the business its always better to use public cloud , if a platform
requires cloud services
\item if a platform requires than public cloud is better choice as it becomes easy to
scale, system is more reliable , no other maintainence, no additional staff
, no extra financial burden to develop own facility buy servers.
\item Also more security in public cloud , bugs  and vulnerability are identified
and patche applied periodically, physically more secured, backups at multiple
locations , easy to scale at a click of a button .
\item proper security check for the cloud admins.
\item Security Assestment and certifications.
\item regulations of government.
\end{itemize}
\end{itemize}


\begin{itemize}
\item Private Cloud Advantage:
\begin{itemize}
\item if business is big enough then it is better to host on private cloud:
\item better end to end control
\item more realiable
\item efficient systems
\item less cost on long runs.
\item improved user experience.
\end{itemize}

\item Cons of Public Cloud :
\begin{itemize}
\item Application may not require thee scalable.
\item May be difficult  require or organize the application to make it ready for
the cloud
\item Even if it is possible to transfer it may not be viable as cost wise, maybe
significant cost.
\item Security is under the cloud responsibility which my be a risk for your
application.
\item Data may being mined
\item Security admin may not be trusted.
\item Physical location of data storage may be in foreign countries ( may not be
firendly to own country).
\item How can it be ensured that deleted is always deleted ?
\item Data breach notification ?
\end{itemize}
\end{itemize}

\subsubsection{Cloud Providers:}
\label{sec:orgc407e19}
\begin{itemize}
\item Amazon Web Services
\item Microsoft Azure
\item Verizon Terremark
\item Google Compute Engine
\end{itemize}

\textbf{*}

\textbf{*}

\subsubsection{Docker Container :}
\label{sec:org5d682d0}
\begin{itemize}
\item A docker is different from the virtual machine that it doesnot contain a
operating system
\end{itemize}

\subsubsection{}
\label{sec:org4e9dad5}
\end{document}
